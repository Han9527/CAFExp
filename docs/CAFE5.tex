

\documentclass{article}
\usepackage[utf8]{inputenc}
\usepackage{bibentry}
\usepackage{filecontents}
\usepackage{listings}
\usepackage{enumitem}
\usepackage{courier}

\lstset{basicstyle=\footnotesize\ttfamily,breaklines=true}
\newcommand{\shortname}{CAFE5 }
\title{CAFE: \textbf{C}omputational \textbf{A}nalysis of gene \textbf{F}amily \textbf{E}volution\\\
Version 5.0\\\
Documentation} 
\author{}

\nobibliography*

\begin{document}
\maketitle
\pagebreak
\tableofcontents
\pagebreak
\section{About CAFE}
CAFE is a software that provides a statistical foundation for evolutionary inferences about changes in gene family size. The program employs a birth and death process to model gene gain and loss across a user-specified phylogenetic tree, thus accounting for the species phylogenetic history. The distribution of family sizes generated under this model can provide a basis for assessing the significance of the observed family size differences among taxa.
\subsection{History}
The original development of the statistical framework and algorithms implemented in CAFE was done by Hahn, et al. and published in \textit{\bibentry{hahn2005estimating}}

CAFE v3.0 was a major update to CAFE v2.1. Major updates in 3.0 included: 1) the ability to correct for genome assembly and annotation error when analyzing gene family evolution using the errormodel command. 2) The ability to estimate separate birth ($\lambda$) and death ($\mu$) rates using the lambdamu command. 3) The ability to estimate error in an input data set with iterative use of the errormodel command using the accompanying python script caferror.py. This version also included the addition of the rootdist command to give the user more control over simulations.

CAFE v4.0 was primarily a maintenance update. At that time formal issue tracking and a user forum was introduced.

The current version is CAFE v5.0. Another major update, \shortname showcases new functionalities while keeping or improving several of the features available in prior releases.

\subsection{New Functionality}
\begin{itemize}
\item{The birth-death model can now estimate the posterior probabilities of each gene family belonging to different evolutionary rate categories. The rates of each of the K categories are determined in similar fashion to what is done in nucleotide sequence analyses: through a discrete gamma distribution that has its alpha parameter estimated by maximum likelihood.}

\item{Ancestral state reconstruction is now done jointly (all ancestral states are inferred simultaneously) with Pupko's algorithm (Pupko et al. 2000).}

\item{The error model is now optimized numerically by \shortname directly, rather than the Python script required in previous versions.}

\item{Outputs are now internally parsed by \shortname (no external dependencies or scripts necessary) into summary tables.}
\end{itemize}
\section{What \shortname does}
\shortname implements a birth-death model for evolutionary inferences about gene family evolution. Its main task is the maximum-likelihood estimation of a global or local gene family evolutionary rates (lambda parameter) for a given data set. Briefly, \shortname can:

\begin{itemize}
\item Compare scenarios in which the whole phylogeny shares the same (global) lambda vs. scenarios in which different parts of the phylogeny share different (local) lambdas;

\item Classify specific gene families as "fastly evolving" in at least two different ways (see documentation below);

\item Infer gene family counts at all internal nodes (ancestral populations) of the phylogeny provided as input. By comparing different nodes in the phylogeny, the user will be able to tell along which branches gene families contracted or expanded;

\item Account for non-biological factors (e.g., genome sequencing and coverage differences, gene family clustering errors, etc.) leading to incorrect gene family counts in input files. This is done with an error model.
\end{itemize}

\section{What CAFE5 does NOT do}
\begin{itemize}
\item{Estimate a phylogeny from gene families or gene sequence alignments. \shortname also does not convert a phylogeny with branches in expected substitutions per site into a time tree (an ultrametric tree with branch lengths in time units). This task should be conducted by the user prior to \shortname analyses.}

\item{Implement clustering algorithms that identify (or verify the legitimacy of) gene families. It is entirely up to the user to pick a gene family identification method and carry out this task prior to \shortname analyses}

\item{Predict gene family function or infer enrichment of functional classes.}
\end{itemize}
\newpage
\section{How to Cite}
If you use \shortname in your work, please cite the application as 

\begin{quotation}
(xxx - a Zenodo DOI to be determined when released)
\end{quotation}
To cite the paper outlining the work underlying \shortname, please use: 

\begin{quotation}
(xxx: Fabio and Matt's paper on the gamma categories)
\end{quotation}

The citation for CAFE v4.0 and v3.1 is
\begin{quotation}
\bibentry{han2013estimating}
\end{quotation}

The citation for CAFE v2.0 is:
\begin{quotation}
\bibentry{hahn2007accelerated}
\end{quotation}

Original development of the statistical framework and algorithms implemented in CAFE are published in:
\begin{quotation}
\bibentry{hahn2005estimating}
\end{quotation}

\section{Installation}
\subsection{Download}
Download the release from the release page
\subsection{Compile}
Move to the CAFE folder, and type
\begin{lstlisting}
$ ./configure
$ make
\end{lstlisting}

The \shortname executable will be put inside the "release" folder.
Then just copy the binary file to somewhere on your path, such as /usr/local/bin. Alternatively, add /path/to/\shortname/release/ to your \$PATH variable (in your .bashrc or .bash profile).

\newpage
\section{Running \shortname}
\shortname performs three different operations on either one or two models. The operations are 
\begin{itemize}
    \item Estimate Lambda - the traditional function of CAFE. Takes a tree and a file of gene family counts, and performs a maximum likelihood calculation to estimate the most likely rate of change across the entire tree.
    \item Reconstruct - Attempts to rebuild a likely count of gene families at each node in the tree. This can be done with an estimated or specified lambda value. To specify a lambda value, pass either the -l or -m parameters with the specified values.
    \item Simulate - Given specified values, generate an artificial list of gene families that matches the values. To generate a simulation, pass the -{}-simulate or -s parameter. Either pass a count of families to be simulated with the parameter (-{}-simulate=1000) or pass a -{}-rootdist (-f) parameter with a file containing the distribution to match (see \ref{rootdist} for the file format).
\end{itemize}
The models are
\begin{itemize}
    \item Base - Perform computations as if no gamma function is available
    \item Gamma - Perform computations as if each gene family can belong to a different evolutionary rate category. To use Gamma modelling, pass the -k parameter specifying the number of categories to use.
\end{itemize}
Unlike earlier versions, \shortname does not require a script. All options are given at once on the command line. Here is an example:
\begin{lstlisting}
cafexp -t examples/tree.txt -i genefamilies.txt -p
\end{lstlisting}

In this example, the -t parameter specifies a file containing the tree that CAFE uses; and the -i parameter specifies a list of gene families. The -p, in this instance given without a parameter, indicates that the root equilibrium frequency will not be a uniform distribution.
\subsection{Parameters}

\begin{description}[leftmargin=1cm, style=nextline]
  \item[-{}-fixed\_alpha, -a] Alpha value of the discrete gamma distribution to use in category calculations. If not specified, the alpha parameter will be estimated by maximum likelihood.
  \item[--error\_model, -e] Error model file path (see \ref{errormodel})
  \item[-{}-rootdist, -f] Root distribution file path  (see \ref{rootdist})
  \item[-{}-infile, -i] character or gene family file path (see \ref{familyfile})
  \item[-{}-n\_gamma\_cats, -k] Number of gamma categories to use. If specified, the Gamma model will be used to run calculations; otherwise the Base model will be used.
  \item[-{}-fixed\_lambda, -l] Single Lambda value
  \item[-{}-fixed\_multiple\_lambdas] Multiple lambda values, comma separated
  \item[-{}-output\_prefix, -o] Output prefix - added to output file names
  \item[-{}-poisson, -p] Use a Poisson distribution for the root frequency distribution. Without specifying this, a normal distribution will be used. A value can be specified (-p10, or -{}-poisson=10); otherwise the distribution will be estimated from the gene families.
  \item[-{}-chisquare\_compare, -r] Chi square compare
  \item[-{}-simulate, -s] Simulate families. Optionally provide the number of simulations to generate (-s100, or -{}-simulate=100)
  \item[-{}-tree, -t] Tree file path - Required for estimation
  \item[-{}-lambda\_tree, -y] Lambda tree file path
\end{description}

\subsection{Input files}
\begin{itemize}
\item{Tree files}

A tree file is specified in Newick format.

\begin{lstlisting}
(A:1,B:1);
\end{lstlisting}
An example may be found in the examples/test\_tree.txt file.

When nodes are mentioned in output files and logs, they are named by the leaf nodes descending from them, in alphabetical order. For example, the parent of A and B in this example will be named "AB". A node containing descendant species of cow, whale, giraffe, and manatee would be named "cowgiraffemanateewhale".

\item{Family files} \label{familyfile}

Family files can be specified in the CAFE input format:

\begin{lstlisting}
Desc    Family ID       A       B
(null)  1       1       2
(null)  2       2       1
(null)  3       3       6
(null)  4       6       3
\end{lstlisting}
The file is tab-separated with a header line giving the order of the species. Each line thereafter consists of a description, a family ID, and counts for each species in the tree.

Alternatively, the family file can be specified with a set of lines beginning with hashtags containing the species order:

\begin{lstlisting}
#A
#B
1       2     1
2       1     2
3       6     3
6       3     4
\end{lstlisting}

In this case, the family ID will be in the final column.

\item{Root distributions} \label{rootdist}

A root distribution file takes the format "family\_size [whitespace] family\_count", e.g.
\begin{lstlisting}
1 1
2 5
3 10
4 15
5 42
\end{lstlisting}
An example may be found in the "examples/poisson\_root\_dist\_1000.txt" file.

\item{Error models} \label{errormodel}

An error model consists of modifications of probabilities of moving from one family size to another through the tree. The file is structured as a series of lines containing the family size, the probability of moving to less than that size, the probability of that size staying the same, and the probabilities of the size becoming larger. Two header lines must be included: the maximum family size to process, and the differential of the probabilities.
\begin{lstlisting}
maxcnt:90
cntdiff -1 0 1
0 0.00 0.95 0.05
1 0.05 0.9 0.05
2 0.05 0.9 0.05
3 0.05 0.9 0.05
4 0.05 0.9 0.05
\end{lstlisting}

\end{itemize}

\subsection{Output}
\begin{itemize}
\item{\_asr.txt}

The file will be named Base\_asr.txt or Gamma\_asr.txt, based on which model is in play. It contains the reconstructed states of the families, in the Nexus file format (https://en.wikipedia.org/wiki/Nexus\_file). A tree is provided for each
family,with the expected family size set off with an underscore from the node ID. 
In the case of a Gamma reconstruction, each category will have its own expected family size, with the overall family size at the end. For example, given three categories, an internal node with the ID of 3 might look like 3\_12\_10\_11\_12. The expected sizes for each of the three categories are 12, 10, and 11 respectively, with an overall weighted expectation of 12. 

In the case of the Gamma reconstruction, the Lambda multipliers for each category are given their own section in this file.

\item{\_family\_results.txt}

The file will be named Base\_family\_results.txt or Gamma\_family\_results.txt, based on which model is in play. It consists of a header line giving the name of each node in the tree, followed by a line consisting of the family ID, an estimate of whether the change is significant('y' or 'n') followed by a series of 'c' (constant), 'i' (increasing), or 'd' (decreasing) showing whether the node is larger, smaller, or consistent with its parent node. The characters are separated by tabs.

\begin{lstlisting}
#FamilyID   pvalue  *   orang   gibbon  chimp   human     
0           0.436   n   d       d       c       c 
1           0.209   n   c       i       c       c
2           0.002   y   c       c       i       i
\end{lstlisting}

In the Gamma model, an additional set of probabilities are appended, representing the likelihood of the family belonging to each gamma category.

\item{\_clade\_results.txt}

The file will be named Base\_clade\_results.txt or Gamma\_clade\_results.txt, based on which model is in play. It consists of a header line, "Taxon Increase Decrease", and for each node in the tree, a tab-separated count of the number of families which have increased and decreased for that node.

\item{\_family\_lks.txt}

Using the Base model, a tab-separated file consisting of the header line "\#FamilyID Likelihood of Family", and additional tab-separated lines consisting of the family ID and the posterior probability of that family.

Using the Gamma model, the file contains the following values:

\begin{itemize}
\item{\#FamilyID}
\item{Gamma Cat Median}
\item{Likelihood of Category}
\item{Likelihood of Family}
\item{Posterior Probability}
\item{Significant}
The values for each family are listed on each tab-separated line.
\end{itemize}

\item{results.txt}

A file giving the name of the model that was selected ("Base" or "Gamma"), the final likelihood of that model, the final value of the rate of change of families (Lambda) that was calculated, and, if an error model was specified, the final value of that value (Epsilon).

\item{simulation\_.txt}

In the case of simulation, a family file (see \ref{familyfile}) is generated with simulated data based on the given input parameters. 

\item{simulation\_truth\_.txt}

When simulating, an additional file is generated with simulated data for internal nodes included. The format is otherwise identical to the family file. Rather than node names, each internal node is assigned an integer ID.

\end{itemize}

\section{Examples}
\subsection{Lambda Search}
Search for a single lambda value using the Newick tree of mammals in the Samples folder, and the family files from the CAFE tutorial:
\begin{lstlisting}
cafexp -t examples/mammals_tree.txt -i data/filtered_cafe_input.txt -p
\end{lstlisting}
In earlier versions, the following script would have returned the same values:
\begin{lstlisting}
tree ((((cat:68.710507,horse:68.710507):4.566782,cow:73.277289):20.722711,(((((chimp:4.444172,human:4.444172):6.682678,orang:11.126850):2.285855,gibbon:13.412706):7.211527,(macaque:4.567240,baboon:4.567240):16.056992):16.060702,marmoset:36.684935):57.315065):38.738021,(rat:36.302445,mouse:36.302445):96.435575)
load -i filtered_cafe_input.txt -t 10 -l reports/run6_caferror_files/cafe_log.txt
lambda -s -t ((((1,1)1,1)1,(((((1,1)1,1)1,1)1,(1,1)1)1,1)1)1,(1,1)1)
\end{lstlisting}
\subsection{Lambda Search with Multiple Lambdas}
Search for multiple lambda values using the Newick tree of mammals in the Samples folder, and the family files from the CAFE tutorial:
\begin{lstlisting}
cafexp -t data/mammals_integral_tree.txt -i data/filtered_cafe_input.txt -p -y examples/mammals_lambda_tree.txt
\end{lstlisting}
In earlier versions, the following script would have returned the same values:
\begin{lstlisting}
((((cat:68,horse:68):4,cow:73):20,(((((chimp:4,human:4):6,orang:11):2,gibbon:13):7,(macaque:4,baboon:4):16):16,marmoset:36):57):38,(rat:36,mouse:36):96) 
load -i integral_test_families.txt -t 10
lambda -s -t ((((1,1)1,1)1,(((((2,2)2,2)2,2)2,(1,1)1)1,1)1)1,(1,1)1)
\end{lstlisting}
\subsection{Error Models}
Search for a single lambda value using the Newick tree of mammals in the Samples folder, and the family files from the CAFE tutorial, applying an error model:
\begin{lstlisting}
cafexp -t data/mammals_integral_tree.txt -i data/filtered_cafe_input.txt -p -l 0.01 -e data/cafe_errormodel_0.0548828125.txt
\end{lstlisting}
In earlier versions, the following script would have returned the same values:
\begin{lstlisting}
tree ((((cat:68,horse:68):4,cow:73):20,(((((chimp:4,human:4):6,orang:11):2,gibbon:13):7,(macaque:4,baboon:4):16):16,marmoset:36):57):38,(rat:36,mouse:36):96) load -i integral_test_families.txt -t 10
errormodel -all -model cafe_errormodel_0.0548828125.txt
lambda -l 0.01 -t ((((1,1)1,1)1,(((((1,1)1,1)1,1)1,(1,1)1)1,1)1)1,(1,1)1) -score
\end{lstlisting}
\subsection{Error Model Estimation}
Estimate an error model:
\begin{lstlisting}
cafexp -t mammals_integral_tree.txt -i filtered_cafe_input.txt -p -e errormodel.txt
\end{lstlisting}
In earlier versions, the following script would have returned the same values:
\begin{lstlisting}
load -i filtered_cafe_input.txt -t 4 -l reports/log_run6.txt
tree ((((cat:68.710507,horse:68.710507):4.566782,cow:73.277289):20.722711,(((((chimp:4.444172,human:4.444172):6.682678,orang:11.126850):2.285855,gibbon:13.412706):7.211527,(macaque:4.567240,baboon:4.567240):16.056992):16.060702,marmoset:36.684935):57.315065):38.738021,(rat:36.302445,mouse:36.302445):96.435575)
lambda -s -t ((((1,1)1,1)1,(((((1,1)1,1)1,1)1,(1,1)1)1,1)1)1,(1,1)1)
report reports/report_run6
\end{lstlisting}

\section{Troubleshooting}
\subsection{Known Limitations}
Because the random birth and death process assumes that each family has at least one gene at the root of the tree, \shortname will not provide accurate results if included gene families were not present in the most recent common ancestor (MRCA) of all taxa in the tree. For example, even if all taxa have a gene family size of 0, CAFE will assign the MRCA a gene family of size 1, and include the family in estimation of the birth and death rate. This difficulty does not affect analyses containing families that go extinct subsequent to the root node.

If a change in gene family size is very large on a single branch, \shortname may fail to provide accurate $\lambda$ estimation and/or die during computation. To see if this is a problem, look at the likelihood scores computed during the $\lambda$ search (reported in the log file if the job finishes). If ALL scores are “-inf” then there is a problem with large size change giving \shortname a probability of 0. Removing the family with the largest difference in size among species and rerunning \shortname should allow $\lambda$ to be estimated on the remaining data. If the problem persists, remove the family with the next largest difference and proceed in a like manner until \shortname no longer finds families with zero probability. However, if rapidly evolving families are removed, care should be taken in interpretation of the estimated average rate of evolution for the remaining data.

In very large phylogenetic trees there can be many independent lambda parameters (2n - 2 in a rooted tree, where n is the number of taxa). \shortname does not always converge to a single global maximum with large numbers of $\lambda$ parameters, and therefore can give misleading results. To check for this you should always run the $\lambda$ search multiple times to ensure that the same estimated values are found. Also, the likelihood of models with more parameters should always be lower than models with fewer parameters, which may not be true if \shortname has failed to find a global maximum. If CAFE does not converge over multiple runs, then one should reduce the number of parameters estimated and try again.

\section{Technical}

\subsection{How does the optimizer work?}

The Nelder-Mead optimization algorithm is used. It runs until it can find a difference of less than 1e-6 in either the calculated score or the calculated value, or for 10,000 iterations. The parameters that are used for the optimizer are as follows:

\begin{itemize}
  \item rho: 1 (reflection)
  \item chi: 2 (expansion)
  \item psi: 0.5 (contraction)
  \item sigma: 0.5 (shrink)
\end{itemize}

In some cases, the optimizer suggests values that cannot be calculated (due to saturation, negative values, or other reasons) In this case, an infinite score is returned and the optimizer continues.

When optimizing for an alpha value with a set number of clusters, if the largest multiplier in the longest branch is saturated, the scorer will return an infinite value.

Certain options are available at compile-time for the optimizer. If OPTIMIZER\_STRATEGY\_INITIAL\_VARIANTS is defined, the optimizer
will take several shorter attempts at various values 
before settling on one value to continue on with. This may cause the optimizer to take more iterations to finish but may have 
greater accuracy. If OPTIMIZER\_STRATEGY\_PERTURB\_WHEN\_CLOSE is defined, the optimizer will begin searching a wider range of values
when it is getting close to a solution. This attempts to get the optimizer out of a local optima it may have found.

\subsection{How does the simulator choose what lambda to use?}

Although the user specifies the lambda, in order to give more
family variety a multiplier is selected every every 50 simulated families. So if 10,000 families are being simulated, 200 different lambdas will be used.

When simulating without the gamma model, the multiplier is a
random value based on a normal distribution
with a mean of 1 and a standard deviation of 0.3.

When simulating WITH the gamma model, the multiplier is drawn 
directly from a gamma distribution based on the selected alpha
and a mean of 1. If clustering is requested via the -k parameter,
the selected cluster multiplier is modified by a normal distribution
with the mean at the value of the multiplier, and a standard
deviation intended to reflect the number of clusters requested.

\section{Acknowledgements}
Many people have contributed to the CAFE project, either by code or ideas. Thanks to:
\begin{itemize}
  \item Mira Han
  \item Gregg Thomas
  \item Ben Fulton
  \item Matthew Hahn
\end{itemize}

\bibliographystyle{plain}
\nobibliography{CAFE5.bib}

\end{document}

